%-------------------------
% Resume in Latex
% Author : Jake Gutierrez
% Based off of: https://github.com/sb2nov/resume
% License : MIT
%------------------------

\documentclass[letterpaper,11pt]{article}

\usepackage{latexsym}
\usepackage[empty]{fullpage}
\usepackage{titlesec}
\usepackage{marvosym}
\usepackage[usenames,dvipsnames]{color}
\usepackage{verbatim}
\usepackage{enumitem}
\usepackage[hidelinks]{hyperref}
\usepackage{fancyhdr}
\usepackage[english]{babel}
\usepackage{tabularx}
\input{glyphtounicode}


%----------FONT OPTIONS----------
% sans-serif
% \usepackage[sfdefault]{FiraSans}
% \usepackage[sfdefault]{roboto}
\usepackage[sfdefault]{noto-sans}
% \usepackage[default]{sourcesanspro}

% serif
% \usepackage{CormorantGaramond}
% \usepackage{charter}


\pagestyle{fancy}
\fancyhf{} % clear all header and footer fields
\fancyfoot{}
\renewcommand{\headrulewidth}{0pt}
\renewcommand{\footrulewidth}{0pt}

% Adjust margins
\addtolength{\oddsidemargin}{-0.5in}
\addtolength{\evensidemargin}{-0.5in}
\addtolength{\textwidth}{1in}
\addtolength{\topmargin}{-.5in}
\addtolength{\textheight}{1.0in}

\urlstyle{same}

\raggedbottom
\raggedright
\setlength{\tabcolsep}{0in}

% Sections formatting
\titleformat{\section}{
  \vspace{-4pt}\scshape\raggedright\large
}{}{0em}{}[\color{black}\titlerule \vspace{-5pt}]

% Ensure that generate pdf is machine readable/ATS parsable
\pdfgentounicode=1

%-------------------------
% Custom commands
\newcommand{\resumeItem}[1]{
  \item\small{
    {#1 \vspace{-2pt}}
  }
}

\newcommand{\resumeSubheading}[4]{
  \vspace{-2pt}\item
    \begin{tabular*}{0.97\textwidth}[t]{l@{\extracolsep{\fill}}r}
      \textbf{#1} & #2 \\
      \textit{\small#3} & \textit{\small #4} \\
    \end{tabular*}\vspace{-7pt}
}

\newcommand{\resumeSubSubheading}[2]{
    \item
    \begin{tabular*}{0.97\textwidth}{l@{\extracolsep{\fill}}r}
      \textit{\small#1} & \textit{\small #2} \\
    \end{tabular*}\vspace{-7pt}
}

\newcommand{\resumeProjectHeading}[2]{
    \item
    \begin{tabular*}{0.97\textwidth}{l@{\extracolsep{\fill}}r}
      \small#1 & #2 \\
    \end{tabular*}\vspace{-7pt}
}

\newcommand{\resumeSubItem}[1]{\resumeItem{#1}\vspace{-4pt}}

\renewcommand\labelitemii{$\vcenter{\hbox{\tiny$\bullet$}}$}

\newcommand{\resumeSubHeadingListStart}{\begin{itemize}[leftmargin=0.15in, label={}]}
\newcommand{\resumeSubHeadingListEnd}{\end{itemize}}
\newcommand{\resumeItemListStart}{\begin{itemize}}
\newcommand{\resumeItemListEnd}{\end{itemize}\vspace{-5pt}}

%-------------------------------------------
%%%%%%  RESUME STARTS HERE  %%%%%%%%%%%%%%%%%%%%%%%%%%%%


\begin{document}

% ----------HEADING----------
% \begin{tabular*}{\textwidth}{l@{\extracolsep{\fill}}r}
%   \textbf{\href{http://sourabhbajaj.com/}{\Large Sourabh Bajaj}} & Email : \href{mailto:sourabh@sourabhbajaj.com}{sourabh@sourabhbajaj.com}\\
%   \href{http://sourabhbajaj.com/}{http://www.sourabhbajaj.com} & Mobile : +1-123-456-7890 \\
% \end{tabular*}

\begin{center}
    \textbf{\Huge \scshape Josh Lin} \\ \vspace{1pt}
    \small 647-891-5329 $|$ \href{mailto:jiexulin99@gmail.com}{\underline{jiexulin99@gmail.com}} $|$ 
    \href{https://linkedin.com/in/jiexulin}{\underline{linkedin.com/in/jiexulin}} $|$
    \href{https://github.com/linj121}{\underline{github.com/linj121}}
\end{center}


%-----------EDUCATION-----------
\section{Education}
  \resumeSubHeadingListStart
    \resumeSubheading
      {McMaster University}{Hamilton, ON}
      {Bachelor of Applied Science in Honours Computer Science}{September 2018 -- April 2023}
  \resumeSubHeadingListEnd

%
%-----------PROGRAMMING SKILLS-----------
\section{Technical Skills}
 \begin{itemize}[leftmargin=0.15in, label={}]
    \small{\item{
      \textbf{Languages}{: JavaScript, TypeScript, CSS, HTML, Java, Python,   SQL, Shell} \\
      \textbf{Web Frontend}{: React, Redux, MUI, Tailwind, Sass, GraphQL,   Webpack, Vite, Jest, Cypress} \\
      \textbf{Web Backend}{: NextJS, NodeJS, Express, Django, Prisma,   PostgreSQL, MongoDB}\\
      \textbf{DevOps}{: Jenkins, Nginx, Docker, Kubernetes(K8s), AWS, GCP} \\
      \textbf{Tools/Others}{: Git, Github, OpenCV, Jira, Excel, Figma,   Postman, FFmpeg} \\
     }}
 \end{itemize}


%-------------------------------------------


%-----------EXPERIENCE-----------
\section{Experience}
  \resumeSubHeadingListStart
    \resumeSubheading
      {Fullstack Web Developer}{July 2023 -- Present}
      {Longan Vision Corp.}{Hamilton, ON}
      \resumeItemListStart
        \resumeItem{Leading a team of 3 in building a live stream platform from scratch, employing the \textbf{Agile} approach, reviewing \textbf{PR} and solving code conflicts using \textbf{Git}, and writing technical documents to guide the team}
        \resumeItem{Conceptualized user-friendly UI designs using Figma, implemented  them using \textbf{React}, \textbf{Redux}, \textbf{MUI}, utilized \textbf{GraphQL} for communication with backend services and established end-to-end tests with \textbf{Cypress}}
        \resumeItem{Mitigated a notable latency issue in HLS streaming by transitioning to \textbf{WebRTC} protocol, reduced the live stream latency from 8s down to \textbf{300ms $\sim$ 1.8s}, which ensured real-time communication}
        \resumeItem{Tackled the scalability challenge in WebRTC peer-to-peer connections by leveraging a media server for handling concurrent streams, significantly reduced client side burden and improved \textbf{system scalability}}
        \resumeItem{Fortified stream security and implemented user authentication by developing a \textbf{REST API} that reacts to the \textbf{Webhook} of the media server, utilizing \textbf{NodeJS}, \textbf{Express}, \textbf{Prisma}, \textbf{MongoDB}, and \textbf{JWT} (Json Web Token)}
        \resumeItem{Reduced system resource consumption by utilizing \textbf{Webhook} for on-demand stream publishing and playing}
        \resumeItem{Integrated object detection into live stream with a latency under 2s, using \textbf{OpenCV}, \textbf{Django} and \textbf{FFmpeg}}
        \resumeItem{Improved development productivity by automating \textbf{CI/CD} pipeline for frontend and multiple backend services using \textbf{Jenkins}, \textbf{Docker Compose} and \textbf{Nginx} on \textbf{GCP} (Google Cloud Platform)}
    \resumeItemListEnd

    \resumeSubheading
      {Data Analyst Intern}{June 2021 -- Jan 2022}
      {Didi Global Inc.}{Beijing}
      \resumeItemListStart
        % 问题:新季度的用户持续流失
        % 目标:挽回用户,提高用户留存率
        % 行动:使用漏斗分析法,通过各个维度,分析新老用户数据
        % 结果:与运营部门合作,使用户留存率提高了百分之5.2
        \resumeItem{Utilized funnel analysis across various dimensions to tackle customer churn, and collaborated with operations department to boost monthly customer retention rate by \textbf{5.2\%} and GMV by \textbf{2.8\%}}
        % 问题:用户活跃度低,用户使用时长短
        % 目标:分析原因,并搭建一套分析体系,便于后续跟踪
        % 行动:定义并搭建用户生命周期模型,协助数据部门搭建数据仓库(DWM),以方便取数分析
        % 结果:提升了分析效率,为后续长期的用户活跃度监控和提升提供了良好的框架和基础
        \resumeItem{Addressed customer inactivity issue by designing a lifecycle framework, aiding the data department in setting up a DWM table, which enhanced analysis \textbf{efficiency} and laid a solid foundation for future analysis}
        % 问题:每日需要手动提取数据,生成报表,发送给各个部门
        % 目标:自动化流程,每日准时提取和发送报表
        % 行动:和RPA部门合作,使用Python编写脚本,实现自动提取下载数据,生成excel报表,并发送到群里
        % 结果:节省了大量时间,提高了工作效率
        \resumeItem{Improved work efficiency and reduced manual work by automating daily \textbf{HiveSQL} data extraction and Excel report generation utilizing \textbf{Python} integrated with chatbot API, which ensured timely report sharing}
      \resumeItemListEnd
      
% -----------Multiple Positions Heading-----------
   % \resumeSubSubheading
   %  {Software Engineer I}{Oct 2014 - Sep 2016}
   %  \resumeItemListStart
   %     \resumeItem{Apache Beam}
   %       {Apache Beam is a unified model for defining both batch and streaming data-parallel processing pipelines}
   %  \resumeItemListEnd
   % \resumeSubHeadingListEnd
% -------------------------------------------

    % \resumeSubheading
    %   {Information Technology Support Specialist}{Sep. 2018 -- Present}
    %   {Southwestern University}{Georgetown, TX}
    %   \resumeItemListStart
    %     \resumeItem{Communicate with managers to set up campus computers used on campus}
    %     \resumeItem{Assess and troubleshoot computer problems brought by students, faculty and staff}
    %     \resumeItem{Maintain upkeep of computers, classroom equipment, and 200 printers across campus}
    % \resumeItemListEnd

    % \resumeSubheading
    %   {Artificial Intelligence Research Assistant}{May 2019 -- July 2019}
    %   {Southwestern University}{Georgetown, TX}
    %   \resumeItemListStart
    %     \resumeItem{Explored methods to generate video game dungeons based off of \emph{The Legend of Zelda}}
    %     \resumeItem{Developed a game in Java to test the generated dungeons}
    %     \resumeItem{Contributed 50K+ lines of code to an established codebase via Git}
    %     \resumeItem{Conducted  a human subject study to determine which video game dungeon generation technique is enjoyable}
    %     \resumeItem{Wrote an 8-page paper and gave multiple presentations on-campus}
    %     \resumeItem{Presented virtually to the World Conference on Computational Intelligence}
    %   \resumeItemListEnd

\resumeSubHeadingListEnd


%-----------PROJECTS-----------
\section{Projects}
    \resumeSubHeadingListStart
      \resumeProjectHeading
          {\textbf{\underline{\href{https://github.com/xRanger-RTMS}{xRangerRtms}}} $|$ \emph{TypeScript, React, MobX, Flask, Django, PostgreSQL, Git}}{January 2023 -- April 2023}
          \resumeItemListStart
            \resumeItem{Collaborated in a team of 4 using \textbf{Agile} methodologies to develop a real-time online monitoring and management system, enabling customers to track robot locations and statuses effectively and efficiently}
            \resumeItem{Conceived a user-friendly frontend interface using \textbf{TypeScript} and \textbf{ReactJS}, streamlining the monitoring of robot status, alerts, and notifications. Amplified UI aesthetics using \textbf{Bootstrap-React, Font Awesome, and SCSS}, and utilized \textbf{MobX} for application state management}
            \resumeItem{Orchestrated a model for maintaining robot data using the \textbf{MVC} pattern, retrieving real-time and historical data from the xRangerTelemetry backend \textbf{REST API} asynchronously}
            \resumeItem{Devised a reference counting mechanism using React \textbf{useEffect} hook to start and stop updates for the robot model based on usage across different components. This strategy decreased http requests, reduced backend server load by \textbf{13\%}, and ensured data consistency across all components}
          \resumeItemListEnd
      % \resumeProjectHeading
      %     {\textbf{\underline{\href{https://superchat-55fd7.firebaseapp.com/}{Superchat}}} $|$ \emph{React.js, JavaScript, HTML/CSS, Firebase}}{December 2022}
      %     \resumeItemListStart
      %       \resumeItem{Accomplished a chat room interface that facilitates real-time message exchanges among users, including message sending, auto-scrolling to the latest message, and real-time updating of chat content}
      %       \resumeItem{Implemented user authentication with Google, using Firebase’s authentication modules, to ensure secure sign-in functionality}
      %       \resumeItem{Leveraged \textbf{Google’s Cloud Firestore} for real-time data handling and storage of chat messages}
      %       \resumeItem{Crafted a dynamic UI that adapts based on the user’s authentication state, incorporating a user-friendly sign-in button and sign-out functionality}
      %     \resumeItemListEnd
    \resumeSubHeadingListEnd
\end{document}
