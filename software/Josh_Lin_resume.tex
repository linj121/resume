%-------------------------
% Resume in Latex
% Template Author : Jake Gutierrez
% Based off of: https://github.com/sb2nov/resume
% License : MIT
%------------------------

\documentclass[letterpaper,11pt]{article}

\usepackage{latexsym}
\usepackage[empty]{fullpage}
\usepackage{titlesec}
\usepackage{marvosym}
\usepackage[usenames,dvipsnames]{color}
\usepackage{verbatim}
\usepackage{enumitem}
\usepackage[hidelinks]{hyperref}
\usepackage{fancyhdr}
\usepackage[english]{babel}
\usepackage{tabularx}
\usepackage{fontawesome} % social icons
\usepackage{geometry} % Margins
\input{glyphtounicode}
\pdfgentounicode=1


%----------FONT OPTIONS----------
% sans-serif
\usepackage[sfdefault]{FiraSans}
% \usepackage[sfdefault]{roboto}
% \usepackage[sfdefault]{noto-sans}
% \usepackage[default]{sourcesanspro}

% serif
% \usepackage{CormorantGaramond}
% \usepackage{charter}

\pagestyle{fancy}
\fancyhf{} % clear all header and footer fields
\fancyfoot{}
\renewcommand{\headrulewidth}{0pt}
\renewcommand{\footrulewidth}{0pt}

% Adjust margins
% \addtolength{\oddsidemargin}{-0.5in}
% \addtolength{\evensidemargin}{-0.5in}
% \addtolength{\textwidth}{1in}
% \addtolength{\topmargin}{-.5in}
\addtolength{\textheight}{1.0in}

% Set margins
\geometry{
  left=0.6in,
  right=0.6in,
  top=0.4in,
  bottom=0.2in
}


\urlstyle{same}

\raggedbottom
\raggedright
\setlength{\tabcolsep}{0in}

% Sections formatting
\titleformat{\section}{
  \vspace{-4pt}\scshape\raggedright\large
}{}{0em}{}[\color{black}\titlerule \vspace{-5pt}]

% Ensure that generate pdf is machine readable/ATS parsable
\pdfgentounicode=1

%-------------------------
% Custom commands
\newcommand{\resumeItem}[1]{
  \item\small{
    {#1 \vspace{-2pt}}
  }
}

\newcommand{\resumeSubheading}[4]{
  \vspace{-2pt}\item
    \begin{tabular*}{0.97\textwidth}[t]{l@{\extracolsep{\fill}}r}
      \textbf{#1} & #2 \\
      \textit{\small#3} & \textit{\small #4} \\
    \end{tabular*}\vspace{-7pt}
}

\newcommand{\resumeSubSubheading}[2]{
    \item
    \begin{tabular*}{0.97\textwidth}{l@{\extracolsep{\fill}}r}
      \textit{\small#1} & \textit{\small #2} \\
    \end{tabular*}\vspace{-7pt}
}

\newcommand{\resumeProjectHeading}[2]{
    \item
    \begin{tabular*}{0.97\textwidth}{l@{\extracolsep{\fill}}r}
      \small#1 & #2 \\
    \end{tabular*}\vspace{-7pt}
}

% \newcommand{\resumeSubItem}[1]{\resumeItem{#1}\vspace{-4pt}}
\newcommand{\resumeSubItem}[1]{
  \begin{itemize}[leftmargin=0.2in, label=\textbullet]
    \item\small{
      {#1 \vspace{-2pt}}
    }
  \end{itemize}
}

\renewcommand\labelitemii{$\vcenter{\hbox{\tiny$\bullet$}}$}

\newcommand{\resumeSubHeadingListStart}{\begin{itemize}[leftmargin=0.15in, label={}]}
\newcommand{\resumeSubHeadingListEnd}{\end{itemize}}
\newcommand{\resumeItemListStart}{\begin{itemize}}
\newcommand{\resumeItemListEnd}{\end{itemize}\vspace{-5pt}}

%-------------------------------------------
%%%%%%  RESUME STARTS HERE  %%%%%%%%%%%%%%%%%%%%%%%%%%%%


\begin{document}

% ----------HEADING----------
% \begin{tabular*}{\textwidth}{l@{\extracolsep{\fill}}r}
%   \textbf{\href{http://sourabhbajaj.com/}{\Large Sourabh Bajaj}} & Email : \href{mailto:sourabh@sourabhbajaj.com}{sourabh@sourabhbajaj.com}\\
%   \href{http://sourabhbajaj.com/}{http://www.sourabhbajaj.com} & Mobile : +1-123-456-7890 \\
% \end{tabular*}

\begin{center}
    \textbf{\Huge \scshape Josh Lin} \\ \vspace{7pt}
    \small \faPhone \thinspace \thinspace 647-891-5329
    $|$ \faEnvelope \thinspace \thinspace \href{mailto:joshlin.dev@gmail.com}{\underline{joshlin.dev@gmail.com}}
    $|$ \faLinkedin \thinspace \thinspace \href{https://linkedin.com/in/jiexulin}{\underline{linkedin.com/in/jiexulin}}
    $|$ \faGithub \thinspace \thinspace \href{https://github.com/linj121}{\underline{github.com/linj121}}
\end{center}


%-----------EDUCATION-----------
\section{Education}
  \resumeSubHeadingListStart
    \resumeSubheading
      {McMaster University}{Hamilton, ON}
      {Bachelor of Applied Science in Honours Computer Science}{September 2018 -- June 2023}
  \resumeSubHeadingListEnd

%

%-----------PROGRAMMING SKILLS-----------
\section{Technical Skills}
 \begin{itemize}[leftmargin=0.15in, label={}]
    \small{\item{
      % TODO: learn and apply: Golang, C++
      \textbf{Languages}{: TypeScript, JavaScript, SQL, HTML/CSS, Python, Java, C++} \\
      % TODO: learn and apply: NextJS, Redux, Tailwind, Styled Components
      \textbf{Frontend}{: React, Redux, Tailwind, Vite, Webpack } \\
      % TODO: learn and apply: SpringBoot
      \textbf{Backend}{: NodeJS, Express, Prisma, PostgreSQL, MongoDB, SQLite, Redis, Jest, Cypress, Selenium, Swagger } \\
      % TODO: learn and apply: Prometheus, Grafana, Terraform, Jenkins, (K8s, ArgoCD?)
      % TODO: get an AWS cert 
      \textbf{CICD/Others}{: Git, GCP, AWS, Docker, GitHub Actions, Nginx, Confluence } \\
     }}
 \end{itemize}

%-----------EXPERIENCE-----------
\section{Experience}
  \resumeSubHeadingListStart

    \resumeSubheading
      {Software Developer}{July 2023 -- July 2024}
      {Longan Vision Corp.}{Hamilton, ON}
        \resumeItemListStart
        % summary 
          \resumeItem{Led the technical redevelopment of a \textbf{live video streaming} platform, improving scalability, real-time streaming capabilities, and UI/UX to meet demands of emergency monitoring applications.}
        \resumeItemListEnd
        % frontend
        \resumeSubSubheading{Frontend}{}
          \resumeItemListStart
            \resumeItem{Reduced webpage load time by 100ms by caching user data in \textbf{session storage}, minimizing repetitive Google Firestore queries, and ensuring fast data retrieval across page reloads.}
            \resumeItem{Designed and implemented dynamic video layout components using \textbf{TypeScript}, \textbf{React} and \textbf{CSS Grid}, supporting grid views with a context menu for channel selection, enhancing monitoring flexibility.}
          \resumeItemListEnd
        % backend/architecture
        \resumeSubSubheading{Backend}{}
          \resumeItemListStart
            \resumeItem{Improved real-time user experience by reducing streaming latency from 8 seconds to under 2 seconds (\textbf{300ms} optimal) using \textbf{WebRTC} as an upgrade to HLS.}
            \resumeItem{Optimized bandwidth and resource consumption for \textbf{IoT} devices by transforming \textbf{WebRTC} from P2P mesh network to a client-server architecture using a C++ media server as a Selective Forwarding Unit.}
            \resumeItem{Secured media resources by implementing user authentication and RBAC through \textbf{RESTful APIs} with token-based validation, using \textbf{Google Firebase SDK}, \textbf{Node.js}, \textbf{Express}, and \textbf{TypeScript}.}
          \resumeItemListEnd
        % devops
        \resumeSubSubheading{Devops}{}
          \resumeItemListStart
            \resumeItem{Simplified the deployment process on Google Cloud Compute Engine by containerizing proxy servers and backend servers with \textbf{Docker Compose}, reducing server setup time by over 80\%.}
            \resumeItem{Secured sensitive data transmission and reduced attack surface by configuring \textbf{Google Cloud VPC} firewall policies and restricting unnecessary port access (HTTPs, SSH and essential TCP/UDP ports).}
            \resumeItem{Reduced application downtime by automating \textbf{TLS/SSL certificate} renewals and developing a Bash script as a post-renewal hook for Certbot to deploy updated certificates.}
          \resumeItemListEnd  

    \resumeSubheading
      {Data Analyst Intern}{June 2021 -- January 2022}
      {Didi Global Inc.}{}
      \resumeItemListStart
        % TODO:
        % 问题:
        % 目标:
        % 行动:
        % 结果: 优化SQL查询
        % \resumeItem{}

        % 问题:新季度的用户持续流失
        % 目标:挽回用户,提高用户留存率
        % 行动:使用漏斗分析法,通过各个维度,分析新老用户数据
        % 结果:与运营部门合作,使用户留存率提高了百分之5.2
        \resumeItem{Boosted monthly customer rentention rate by 5.2\% by identifying key churn areas using funnel analysis on data from both new and existing customers.}
        % 问题:用户活跃度低,用户使用时长短
        % 目标:分析原因,并搭建一套分析体系,便于后续跟踪
        % 行动:定义并搭建用户生命周期模型,协助数据部门搭建数据仓库(DWM),以方便取数分析
        % 结果:提升了分析效率,为后续长期的用户活跃度监控和提升提供了良好的框架和基础
        % \resumeItem{Addressed customer inactivity issue by designing a lifecycle framework, aiding the data department in setting up a DWM table, which enhanced analysis \textbf{efficiency} and laid a solid foundation for future analysis}
        % 问题:每日需要手动提取数据,生成报表,发送给各个部门
        % 目标:自动化流程,每日准时提取和发送报表
        % 行动:和RPA部门合作,使用Python编写脚本,实现自动提取下载数据,生成excel报表,并发送到群里
        % 结果:节省了大量时间,提高了工作效率
        \resumeItem{Automated daily \textbf{HiveSQL} queries and Excel report generation using \textbf{Python}, ensuring timely sharing of insights with team members and stakeholders.}
      \resumeItemListEnd
      
% -----------Multiple Positions Heading-----------
   % \resumeSubSubheading
   %  {Software Engineer I}{Oct 2014 - Sep 2016}
   %  \resumeItemListStart
   %     \resumeItem{Apache Beam}
   %       {Apache Beam is a unified model for defining both batch and streaming data-parallel processing pipelines}
   %  \resumeItemListEnd
   % \resumeSubHeadingListEnd
% -------------------------------------------

    % \resumeSubheading
    %   {Information Technology Support Specialist}{Sep. 2018 -- Present}
    %   {Southwestern University}{Georgetown, TX}
    %   \resumeItemListStart
    %     \resumeItem{Communicate with managers to set up campus computers used on campus}
    %     \resumeItem{Assess and troubleshoot computer problems brought by students, faculty and staff}
    %     \resumeItem{Maintain upkeep of computers, classroom equipment, and 200 printers across campus}
    % \resumeItemListEnd

    % \resumeSubheading
    %   {Artificial Intelligence Research Assistant}{May 2019 -- July 2019}
    %   {Southwestern University}{Georgetown, TX}
    %   \resumeItemListStart
    %     \resumeItem{Explored methods to generate video game dungeons based off of \emph{The Legend of Zelda}}
    %     \resumeItem{Developed a game in Java to test the generated dungeons}
    %     \resumeItem{Contributed 50K+ lines of code to an established codebase via Git}
    %     \resumeItem{Conducted  a human subject study to determine which video game dungeon generation technique is enjoyable}
    %     \resumeItem{Wrote an 8-page paper and gave multiple presentations on-campus}
    %     \resumeItem{Presented virtually to the World Conference on Computational Intelligence}
    %   \resumeItemListEnd

\resumeSubHeadingListEnd


%-----------PROJECTS-----------
\section{Projects}
    \resumeSubHeadingListStart
      \resumeProjectHeading
        {\textbf{Convo, a SaaS platform for messaging apps} $|$ \faGithub \thinspace \thinspace \href{https://github.com/linj121/convo}{\underline{github.com/linj121/convo}}}{July 2024 - Now}
        \resumeItemListStart
          \resumeItem{Implemented a highly customizable plugin system and a task scheduler system for messaging apps (eg. WhatsApp) using \textbf{NodeJS}, \textbf{TypeScript} and \textbf{Prisma} (SQLite3).}
          \resumeItem{Improved user experience by integrating AI Agents with text2speech/speech2text features via \textbf{OpenAI APIs}.}
        \resumeItemListEnd
      
      \resumeProjectHeading
        {\textbf{Self-hosting} $|$ Docker, Grafana, Prometheus, Cloudflare}{June 2024 - Now}
        \resumeItemListStart
          \resumeItem{Self-hosted 11 \textbf{Docker} containers on a cloud VPS for services like Gitea, Mailserver, and Keycloak.}
          \resumeItem{Enhanced email deliverability by authenticating the mail domain with SPF, DKIM and DMARC on \textbf{Cloudflare}.}
          \resumeItem{Maintained high service availability by enabling monitoring and alerts using \textbf{Grafana} and \textbf{Prometheus}.}
        \resumeItemListEnd

    \resumeSubHeadingListEnd

%-----------OPEN SOURCE-----------
% \section{Open Source}
%   \resumeSubHeadingListStart
%     \resumeProjectHeading
%       {\textbf{\underline{\href{https://www.npmjs.com/package/gpt-tokens}{gpt-tokens}}} $|$ OpenAI API tokens consumption and pricing calculator $|$ \emph  {NodeJS, TypeScript}}{}
%       \resumeItemListStart
%         % \resumeItem{Contributed to gpt-tokens, an open source library published on npm (95    stars, 6k weekly downloads) for calculating token and pricing usage of calling  OpenAI  API endpoints}
%         % \resumeItem{Enhanced collaboration and increased productivity by setting up CI    pipeline for unit tests and regression tests}
%         \resumeItem{\faGithub \thinspace \thinspace Pull Request: \underline{github.com/Cainier/gpt-tokens/pull/50} $|$ Maintained stability by developing performance benchmarks for components utilizing tokenization algorithms and refactoring parts of the code to enhance library encapsulation.}
%       \resumeItemListEnd
%   \resumeSubHeadingListEnd




%-------------------------------------------
\end{document}

